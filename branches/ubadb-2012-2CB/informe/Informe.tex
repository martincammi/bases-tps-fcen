\documentclass[11pt, a4paper, spanish]{article}

%%%%%%%%%% COMIENZO DEL PREAMBULO %%%%%%%%%%

%Info sobre este documento
\author{Martin Cammi}
\title{Trabajo Pr'actico de Ingenier'ia del software I}

%\usepackage{infostyle}                                                  % provee un look & feel similar a un documento Word
\usepackage[top=2.5cm, bottom=2.5cm, left=2.5cm, right=2.5cm]{geometry}  % m\'argenes
\usepackage[ansinew]{inputenc}                                           % permite que los acentos del estilo \'a\'e\'i\'o\'u salgan joya
\usepackage[spanish, activeacute]{babel}                                 % idioma espa\~{n}ol, acentos f\'aciles y deletreo de palabras
\usepackage{indentfirst}                                                 % permite indentar un parrafo a mano
\usepackage{caratula}                                                    % incluye caratula est\'andar
\usepackage{graphicx}                                                    % permite insertar gr\'aficos
\usepackage{color}                                                       % permite el uso de colores en el documento
\usepackage[pdfcreator={TexLive!, LaTeX2e con TeXnicCenter y la inteligencia de Jonathan ;-)},
			pdfauthor={Grupo 2"},
			pdftitle={Base de Datos - Trabajo practico:},
			pdfsubject={Trabajo Practico de Buffer Manager},
			pdfkeywords={MER, MR},
			pdfstartview=FitH,            % Fits the width of the page to the window
			bookmarksnumbered,            % los bookmarks numerados se ven mejor...
			colorlinks,                   % links con bellos colores
			linkcolor=magenta]            % permite cambiar el color de los links
			{hyperref}                    % Permite jugar con algunas cosas que aparecer\'an en el PDF final
\usepackage{hyperref}
\usepackage{rotating}
\usepackage{ulem}
\usepackage[dash]{dashundergaps}


%\selectlanguage{spanish}

\linespread{1.3}                    % interlineado equivalente al 1.5 l\'ineas de Word...
\pagestyle{myheadings}              %encabezado personalizable con \markboth{}{}
\markboth{}{Trabajo pr\'actico 2 (Cammi, De Sousa, M\'endez, Serapio) }
\headsep = 30pt                     % separaci\'on entre encabezado y comienzo del p\'arrafo

%\addtolength{\oddsidemargin}{-2cm}	% configuracion IDEAL!!!
%\addtolength{\textwidth}{4cm}
%\addtolength{\textheight}{2cm}

% macro 'todo' para To-Do's
\def\todo#1{\textcolor{red}{#1}}

% Macro 'borde' para un texto con borde
\newsavebox{\fmbox}
\newenvironment{borde}[1]
{\begin{lrbox}{\fmbox}\begin{minipage}{#1}}
{\end{minipage}\end{lrbox}\fbox{\usebox{\fmbox}}\\[10pt]}

%%%%%%%%%% FIN DEL PREAMBULO %%%%%%%%%%

\begin{document}

\materia{Base de Datos}
\submateria{Segundo Cuatrimestre de 2012}
\titulo{Trabajo pr\'actico 2}
\subtitulo{Buffer Manager y estrategias de reemplazo de p'aginas}
\grupo{Grupo 2}

\integrante{Cammi, Mart\'in}{676/02}{martincammi@gmail.com}
\integrante{De Sousa, Mariano}{389/08}{marian\_sabianaa@hotmail.com}
\integrante{M\'endez, Gonz\'alo}{843/04}{gemm83@hotmail.com}
\integrante{Serapio, Noelia}{871/03}{noeliaserapio@gmail.com}

\maketitle

\thispagestyle{empty}

\tableofcontents

\newpage

% Conviene poner las secciones como diferentes archivos,
% sobre todo cuando se trabaja en equipo.
% Es m\'as f\'acil para sincronizar mediante control de versiones.
%\input{Introducci\'on}


% BEGIN Ejemplos de uso

	%\section{Una secci\'on}
	%\label{sec:unaSeccion}
	%Hola! Soy una Secci\'on
	%	\subsection{Una subsecci\'on}
	%		Y yo soy una subsecci\'on!!!
	%		\subsubsection{Una subsubsecci\'on}
	%			Y yo soy una sub-subsecci\'on!!!
	%			\paragraph{Un p\'arrafo\\}
	%				Y yo soy un p\'arrafo, porque no hay mas sub-sub-sub-subsecciones!!!

	%\section{Otra secci\'on}
	%	Como pudimos ver en la secci\'on \ref{sec:unaSeccion}, esto es una demo de una referencia a una secci\'on.
	
	%	Tambi\'en podemos hacer referencia a la p\'agina de la secci\'on:\\[10pt]
	
		% Ejemplo de uso de un borde (falta pulir para que no tire un warning!)
	%	\begin{borde}{0.98\textwidth}
	%		En la p\'agina \pageref{sec:unaSeccion}, hay una secci\'on pilla...
	%	\end{borde}

% END Ejemplos de uso

\newpage 
\section{Investigaci'on sobre estrategia de reemplazo de p'aginas de Oracle}

\subsection{Introducci'on:}

Inicialmente Oracle comenz'o con LRU (Least Recently Used) algoritmo. Sin embargo, este algoritmo estaba teniendo algunos problemas. Por ejemplo, si la cache del buffer tiene 500 bloques y escaneo completo de una tabla est'a recibiendo 600 bloques en cache del buffer, todos los bloques populares desaparecer'an. \\
Para superar este problema, Oracle propuso un algoritmo modificado de LRU. 
   
\subsection{Descripci\'on del algoritmo TouchCounter}

\subsection{Problem\'atica que resuelve}

\subsection{Ventajas y desventajas}

\newpage 
\section{Implementaci'on estrategias de reemplazo de p'aginas}

\subsection{Algoritmo de LRU}

\subsection{Algoritmo de MRU}

\subsection{Algoritmo de Touch Count}

\section{Test de Unidad}

\section{Comparaci\'on de Touch Counter}


\end{document}
